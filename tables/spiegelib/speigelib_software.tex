\begin{table*}[t]
\centering\small
\begin{threeparttable}
	\caption{Algorithms currently implemented in \mintinline{python}{spiegelib}}
	\label{table:spiegel_algorithms}
	\def\arraystretch{1.125}
	\begin{tabular}{|l|l|l|}
	\hline
	\multicolumn{3}{|c|}{\textbf{Algorithms in \mintinline{python}{spiegelib}}} \\
	\hline
	\hline
	\textbf{Audio Representations} & \textbf{Deep Learning Estimators} & \textbf{Search Estimators} \\
	\hline
	FFT 		& MLP \cite{yee2018automatic} 		& Basic GA   			\\
	STFT 		& LSTM/LSTM++ \cite{yee2018automatic} 		& NSGA III \cite{tatar2016automatic}				\\\cline{3-3}
	Mel-Spectrogram		& Conv6/5 \cite{barkan2019inversynth} 	& \textbf{Objective Evaluation} 	\\\cline{3-3}
	MFCC	& Conv6s/5s\tnote{2} 	& MFCC Error \cite{yee2018automatic} 		\\\cline{2-2}
	Spectral\tnote{1} & \textbf{Hybrid Estimators} & LSD \cite{masudo2021quality} \\\cline{2-2}
	& WS-NSGA\tnote{3} & Parameter Error \cite{barkan2019inversynth} \\
	\hline
	\end{tabular}
	\begin{tablenotes}[para, flushleft]
			\footnotesize
			\item[1] Spectral bandwidth, centroid, contrast, flatness, and rolloff.\\
			\item[2] Derivatives of the models from \cite{barkan2019inversynth} with reduced capacity \\
			\item[3] Warm-start NSGA. A novel approach that uses a pre-trained deep learning model with a NSGA-III. Introduced in chapter \ref{chapter:inverse_synth_experiment} of this thesis.
	\end{tablenotes}
\end{threeparttable}
\end{table*}