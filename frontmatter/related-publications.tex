\newpage
\TOCadd{Related Publications}

\chapter*{Related Publications}

A portion of the work presented as a part of this thesis has been published and/or presented at international conferences and venues. 

\subsubsection{Chapter \ref{chapter:spiegelib}}
Chapter \ref{chapter:spiegelib} is based on a conference paper presented at the Audio Engineering Society (AES) Virtual Vienna 2020 conference. This paper was entitled "SpiegeLib: An automatic synthesizer programming library" and was co-authored by the author's supervisors George Tzanetakis and Kirk McNally, who provided feedback with the development of the software library and helped during editing of the resulting paper \cite{shier2020spiegelib}. Chapter \ref{chapter:spiegelib} presents the details of the open-source software library (SpiegeLib) that was the main contribution of the that work. 

\subsubsection{Chapter \ref{chapter:inverse_synth_experiment}}
Chapter \ref{chapter:inverse_synth_experiment} builds off of the example experiment that was included as a component of "SpiegeLib: An automatic synthesizer programming library". This experiment has since been extended and a novel hybrid approach added, which is documented in detail in this thesis. An in-progress version of this experiment that focused on convolutional neural networks for inverse synthesis was presented at the 2020 AES Virtual Symposium: Applications of Machine Learning in Audio. 

\subsubsection{Chapter \ref{chapter:torchsynth}}
Chapter \ref{chapter:torchsynth} is a version of a paper entitled "One Billion Audio Sounds from GPU-enabled Modular Synthesis" that was published in the Proceedings of the 23rd International Conference on Digital Audio Effects (DAFx20in21) \cite{turian2021one}. It should be noted that the author of this thesis was the second author of this paper and was noted as contributing equally to the first author, Dr. Joseph Turian. The other co-authors of this paper were George Tzanetakis, Kirk McNally, and Max Henry. Joseph Turian held the overall vision the research project, contributed to the design and development of synth1B1 and torchsynth, designed and conducted the main research experiments, and lead the writing of the paper. The author of this thesis contributed to the design and development of the synth1B1 dataset and the torchsynth synthesizer, helped design and conduct experiments, helped write the paper, and presented the work at the DAFx conference. Max Henry contributed to the design and development of synth1B1 torchsynth, helped design the experiments, and helped write/edit the paper and provided feedback on the presentation. George Tzanetakis and Kirk McNally provided feedback on the experimental design, participated in informal listening experiments, and helped edit the paper.

\subsubsection{Chapter \ref{chapter:synth-explore}}
Chapter \ref{chapter:synth-explore} builds off of work that the author conducted at the end of his undergraduate degree and completed during the first year of his master's. This work, entitled "Manifold Learning Methods for Visualization and Browsing of Drum Machine Samples" was published in 2021 in the Journal of the Audio Engineering Society (JAES) \cite{shier2021manifold}. It focuses on the development of techniques for visualizing drum samples in two-dimensions using audio feature extraction and manifold learning, and builds off of the authors previously published work on the topic \cite{shier2017analysis, shier2017sieve}. It was co-authored with Kirk McNally, George Tzanetakis, and Ky Grace Brooks. For a summary of this work please refer to appendix \ref{appendix:manifold}. Chapter \ref{chapter:synth-explore} of this thesis draws from this work and uses a similar technique to visualize synthesizer sounds in two-dimensions based on sound similarity.

