\newpage
\TOCadd{Abstract}

\chapter*{Abstract}
The sound synthesizer is an electronic musical instrument that has become commonplace in audio production for music, film, television and video games. Despite its widespread use, creating new sounds on a synthesizer -- referred to as synthesizer programming -- is a complex task that can impede the creative process. The primary aim of this thesis is to support the development of techniques to assist synthesizer users to more easily achieve their creative goals. One of the main focuses is the development and evaluation of algorithms for inverse synthesis, a technique that involves the prediction of synthesizer parameters to match a target sound. Deep learning and evolutionary programming techniques are compared on a baseline FM synthesis problem and a novel hybrid approach is presented that produces high quality results in less than half the computation time of a state-of-the-art genetic algorithm. Another focus is the development of intuitive user interfaces that encourage novice users to engage with synthesizers and learn the relationship between synthesizer parameters and the associated auditory result. To this end, a novel interface (Synth Explorer) is introduced that uses a visual representation of synthesizer sounds on a two-dimensional layout. An additional focus of this thesis is to support further research in automatic synthesizer programming. An open-source library (SpiegeLib) has been developed to support reproducibility, sharing, and evaluation of techniques for inverse synthesis. Additionally, a large-scale dataset of one billion sounds paired with synthesizer parameters (synth1B1) and a GPU-enabled modular synthesizer (torchsynth) are also introduced to support further exploration of the complex relationship between synthesizer parameters and auditory results.

% The experiments conducted as a part of this thesis evaluate several approaches to inverse synthesis, including genetic algorithms and deep learning models, that have been used in previous work. A novel approach is presented that produces results that are as good as the best method evaluated, in less than half the computation time. 



% The primary focus of this thesis is developing algorithms for inverse synthesis and designing user 


% Inverse synthesis involves the use of an algorithm to predict synthesizer parameters to match a target sound and is one of the main focuses of this thesis. The experiments conducted as a part of this thesis evaluate several approaches to inverse synthesis, including genetic algorithms and deep learning models, that have been used in previous work. A novel approach is presented that produces results that are as good as the best method evaluated, in less than half the computation time.  


% Chapter \ref{chapter:spiegelib} describes an open source library (named SpiegeLib) that was developed as a part of this thesis to support the sharing and evaluation of methods for inverse synthesis. C



% Automatic synthesizer programming is the field of research that developed to address the challenges with synthesizer programming and has the goal of creating more intuitive methods of programming sounds into a synthesizer to support the creative goals of synthesizer users. The work presented as a part of this thesis shares this goal and contributes to the field of automatic synthesizer programming in a number of ways.
 
%  The main aim of this thesis is to support the development of techniques and approaches that assist synthesizer users to more effectively achieve their creative goals.


% Sound synthesizers are electronic musical instruments that generate synthetic sounds. They were popularized through the last half of the 1900s and the sounds they create have permeated through almost all genres of music and are commonplace in soundtracks for film, television, and video games. Anyone with a laptop or a mobile phone has the power to make sounds with a synthesizer. Despite their widespread availability, synthesizers are challenging to use. Creating a new sound on a synthesizer -- referred to as synthesizer programming -- is a complex task that involves the manually adjustment of a potentially large number of parameters that are labelled using technical names. This process is further complicated by the fact that modifications to these parameters often do not lead to intuitive auditory results. This disconnect between synthesizer parameters and the associated sound means that there is a large learning curve involved in using a synthesizer effectively. Automatic synthesizer programming is the field of research that developed to address the challenges with synthesizer programming and has the goal of creating more intuitive methods of programming sounds into a synthesizer to support the creative goals of synthesizer users. The work presented in this thesis shares this goal and contributes to the field of automatic synthesizer programming in a number of ways. 



% Despite their widespread use, the task of programming new sounds into a synthesizer is complex and requires a thorough understanding of sound design principles and technical details.

% Sound synthesizers are electronic musical instruments that generate synthetic sounds. Through the last half of the 1900s their popularity grew as more affordable and accessible synthesizers were developed, supported by the development of digital technology and personal computers. The sounds of synthesizers have permeated through almost all genres of music and are commonplace in soundtracks for films, television, and games. Anyone with a laptop or a mobile phone has the power to make sounds with a synthesizer. However, anyone who has tried to program a new sound into a synthesizer will attest to the fact that the usability of these devices has yet to catch up to their availability. 

% The main difficulty with working with synthesizers is that

% Automatic synthesizer programming has been an active field of research since the rise in popularity in synthesizers in the late 1970s and has developed to address the challenges with synthesizer programming.