\chapter{JAES Publication Summary}
\label{appendix:manifold}

A brief summary/abstract is provided here of the author's paper entitled "Manifold Learning Methods for Visualization and Browsing of Drum Machine Samples", which was published in 2021 in the Journal of the Audio Engineering Society (JAES) \cite{shier2021manifold} and co-authored by Kirk McNally, George Tzanetakis, and Ky Grace Brooks.

The use of electronic drum samples is widespread in contemporary music productions, with music producers having an unprecedented number of samples available to them. The task of organizing and selecting from these large collections can be challenging and time-consuming, which points to the need for improved methods for user interaction. This paper presents a system that computationally characterizes and organizes drum machine samples in two-dimensions based on sound similarity. The goal of the work is to support the development of intuitive drum sample browsing systems. The methodology presented explores time segmentation, which isolates temporal subsets from the input signal prior to audio feature extraction, as a technique for improving similarity calculations. Manifold learning techniques are compared and evaluated for dimensionality reduction tasks, and used to organize and visualize audio collections in two-dimensions. This methodology is evaluated using a combination of objective and subjective methods including audio classification tasks and a user listening study. Results show that using time-segmentation lead to overall higher correlations with subjective rankings and that using MDS dimensionality reduction with time-segmentation lead to higher correlations with subjective rankings for similarities in two-dimensions. Finally, we present an open-source audio plug-in developed using the JUCE software framework that incorporates the findings from this study into an application that can be used in the context of a music production environment.