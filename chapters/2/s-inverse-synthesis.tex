\section{Inverse Synthesis}
- inverse synthesis is the problem of estimating parameters for a synthesizer to match a target sound.
- This is the main issue that has been explored in the literature and is a big component of the research that is conducted here
 
This section provides a brief summary of the main algorithmic methodologies that have been used in previous ASP research, namely, optimization and deep learning techniques. Other methods that have been used in ASP research that are beyond the scope of this paper include  include fuzzy logic \cite{mitchell2005frequency, hamadicharef2012intelligent}, linear coding \cite{mintz2007toward}, and query approaches \cite{mcartwright2014}.

\subsection{Search vs. Modelling}
% Add something abouth the search vs. modelling approach
These two methods, the hill-climber and the LSTM++, represent two different methods for inverse synthesis; the hill-climber is a search algorithm and the LSTM++ is a modelling algorithm. Search algorithms (which include genetic algorithms) have been used extensively in the body of automatic synthesizer programming research and modelling with deep learning has been becoming more popular. Search algorithms are presented with a target sound and then begin an iterative search for the parameter settings, attempting to move closer to the target at each iteration.

\subsection{Optimization}
The optimization approach was first introduced in 1993 with Horner et al.'s work on sound matching for FM synthesis using genetic algorithms \cite{horner1993machine}. A genetic algorithm (GA) is a method for solving an optimization problem using techniques based on the principles of Darwinian evolution, and is part of a broader class of evolutionary algorithms \cite{whitley1994genetic}. In a GA, a potential solution (an individual) is represented as an array of bits. An initial set of individuals is randomly generated, and then iteratively evolved using biologically inspired processes including selection, breeding, and mutation. Individuals are ranked using an evaluation function that measures the $fitness$ of a given solution. The objective of a GA is to minimize that value (or maximize it, depending on the problem definition). The best candidates are selected for further evolution until either an optimal solution is found or a set number of evolutions has been completed.

In the case of sound matching, the \textit{fitness} of a potential solution is determined by measuring the error between a target sound and a candidate. Typically, an audio transform or audio feature extraction is performed prior to calculating \textit{fitness}. The first works on synthesizer sound matching with GAs used the Short Time Fourier Transform (STFT) in the evaluation function \cite{horner1993machine, horner1995wavetable}. Mel-frequency Cepstral Coefficients (MFCCs), an audio representation using a non-linear frequency scaling that is more relevant to human hearing, have also been used \cite{yee2008synthbot, roth2011comparison, macret2014automatic, smith2017play}. Tatar et al. introduced the use of a multi-objective GA for synthesizer sound matching that used three different methods for calculating $fitness$ values: the STFT, Fast Fourier Transform (FFT), and signal envelope \cite{tatar2016automatic}. Alternatives to GAs that have been used for sound matching include Particle Swarm Optimization (PSO) \cite{heise2009automatic} and Hill-Climbing \cite{roth2011comparison, luke2019stochastic}.

\subsection{Deep Learning}
Deep learning is subset of machine learning that utilizes artificial neural networks to learn patterns in data and make predictions based on those patterns \cite{lecun2015deep}. Deep learning architectures contain multiple layers comprised of simple non-linear modules. Through iterative training, the layers are able to extract features from raw input data and learn intricate patterns in high-dimensional data. These multi-layer architectures have enabled deep learning models to excel at complex tasks including image recognition, speech recognition, and music related tasks such as audio source separation \cite{spleeter2019}.

In the context of an ASP sound matching experiment, a deep learning model accepts an audio signal as input and predicts synthesizer parameter settings to replicate that audio signal. Audio signals are often preprocessed using audio feature extraction or an audio transform. Models are trained using a large set of example sounds generated from a synthesizer and use the parameter settings that generated a particular sound as the ground truth. During training, the error between predicted parameter settings and the actual parameter settings (the ground truth) are used to evaluate how well the model is learning and to iteratively update variables within the model to improve performance. 

Several researchers have explored the use of deep learning for ASP. Yee-King et al. reviewed several deep learning architectures for FM synthesizer sound matcing [11]. In their work, they compared multi-layer perceptron (MLP), Long Short Term Memory (LSTM), and LSTM++ networks. Barkan et al. explored sound matching using convolutional neural networks (CNNs) [12]. They framed the problem as an image classification task and used the STFT to create spectrogram images of target sounds to use as input to the CNNs. Esling et al. recently presented a novel application called $FlowSynth$ that uses a generative model based on Variational Auto-Encoders and Normalizing Flows [13]. In addition to performing well on sound matching tasks, they also showed that their approach supported novel interactions including macro-control of synthesizer parameters.

\subsection{Evaluation Methods}
% Maybe? A brief review of qualitative vs. quantiative methods?