\section{Audio Synthesizers}
In the context of audio, a synthesizer refers to a device that generates sound or music. Martin Russ provides a thorough overview of audio synthesis in his textbook, Sound Synthesis and Sampling \cite{russ2012sound}. Russ introduces the synthesizer as any device that gerates sound,  even the human voice can be thought of as a synthesizer, however, sound synthesizers have become more broadly accepted as an electronic device that produces synthetic sounds. A synthesizer may do this through the playback and recombination of pre-existing audio material or through the generation of raw audio waveforms. There are numerous types of synthesis techniques that are capable of producing a huge variety of different sounds. Broadly speaking, these sounds can be categorized into two different classes: `imitative` or `synthetic`. Imitative sounds attempt to emulate a sound that exists in the natural world such as a physical musical instrument or a sound effect such as an explosion. Electric pianos are examples of imitative synthesizers. Synthetic sounds are those that have no relation to a sound in the physical world. The distinction between imitative and synthetic sounds is blurry and most sounds fall somewhere in the middle. For example, synth-brass sounds, which were a staple of synths such as Roland's popular Jupiter-8, are sounds that are based on an imitation of a brass sound but extend into the synthetic realm. 

% This is maybe a bit more like introduction material?
Synthesizers have become ubiquitous in audio production and entire genres of music have been developed around their use. Bebe and Louis Barron produced the first electronic film score for the movie "Forbidden Planet" in 1955 using synthetic sounds. Since then the use of synthesizers for movies and video games has also become common-place.


\subsection{Evolution of Synthesizers}
A brief history of synthesizers is provided here for historical context to the development and use of synthesizers. The full history of the topic is beyond the scope of this thesis and the interested reader is referred to some great textbooks that cover audio synthesizers and their history \cite{roads1980interview, mcguire2015musical, jenkins2019analog, russ2012sound}, which were used in the writing of this chapter.
\subsubsection{Analog}
Until the late 1950s all synthesizers were analog. All analog synthesizers are defined by their use of continuous-time signals, as opposed to digital systems which operate in discrete chunks of information, or discrete-time signals. Early analog synthesizers can be broken down into two broad categories: (1) sounds that are generated directly by electric circuits by oscillating vaccuum tubes, or (2) rotating or vibrating physical systems that are controlled by electronic sources \cite{roads1996computer}. The first, as well as the largest sound synthesizer ever built was developed in the early 1900s by Thaddeus Cahill. On September 26, 1906, an audience of 900 individuals came to view the massive electronic instrument called the Telharmonium that was capable of producing pure sinusoidal waves in at frequencies in integer ratios. Other early synthesizers include the Theremin, built by Leon Theremin in 1920, which produced a pure tone with a varying pitch and amplitude that was controlled by a performer moving their arms in relation to two antennas. The sound produced by the theremin has an eery feel to it, well-suited for sci-fi film scores. Versions of the theremin have been used in popular music by musicians including The Beach Boys, Led Zeppelin, and The Rolling Stones (citations? check out wikpedia).  The Ondes Martenot, developed in 1928 by Maurice Martenot and Ondes, had a similar sound to the theremin and also was one of the first synthesizers to include a piano-like keyboard interface.

In the 1960s two companies emerged on opposite sides of America and released synthesizers that have shaped the landscape of audio synthesis. Around 1964 Don Buchla, who lived in the San Francisco area, released the the Buchla 100 Series Modular Electronic Music System and Robert Moog, who lived in the New York area, released the R.A. Moog Modular System \cite{mcguire2015musical}. Both synthesizers were modular systems that had individual processing units called \textit{modules} that could be interconnected using patch cables, connecting together synthesizer modules is known as creating a "synth patch", or simply a "patch". Both systems also include Voltage-Controlled Oscillators (VCOs) which create electronic waveforms at stable musical pitches, and the pitch can be controlled via an input signal. The Moog Modular System also featured a Voltage Controlled Filter (VCF) that was an early version of Moog's famous ladder filter design that would resonate at a controllable center frequency, creating some of the most iconic synthesizer sounds that are still used today.

There were some important philosophical differences between Moog and Buchla Synthesizers that have lead to two different schools of thought: East Coast and West Coast synthesis. The development of the Buchla synthesizer by Don Buchla was guided by Morton Subotnick and other experimental composers working out the San Francisco Tape Music Center. Subotnick explicitly requested that the synthesizer was not controlled by a traditional keyboard interface as he was worried that it would trap him into created tonal music. Instead, Buchla synthesizers are controlled using a set of touch plates and sequencers. At the same time on East coast Robert Moog was developing the Moog Modular and building a traditional keyboard interface to control the pitch of his VCOs. This feature, which allowed the Moog Modular Systems to be integrated more easily with Western music, was one of the reasons that Moog synthesizers became much more popular and commercially successful compared to Buchla. 

Another reason that Moog synthesizers were launched into the public eye is because of their use in recorded music. In 1968 Wendy Carlos used a Moog Modular synthesizer to orchestrate, perform, and record a selection of Johann Sebastian Bach pieces. The collection of music, called \textit{Switched On Bach}, went on to become the best selling classical recording of all time. Other musicians had created recreations of classical music pieces on synthesizers, but none had reached the same level as \textit{Switch On Bach}. Jenkins \cite{jenkins2019analog} credits the nature of the counterpoint in Bach's pieces as being particularly well suited to synthesizers, as well as Carlos' ability to design sounds that brought that performance alive, to the success of the release. Building on this success Carlos went on to score synthesized soundtracks for movies including Stanley Kubrick's \textit{A Clockwork Orange}. Another exceptional example of classical music recreated using Moog synthesizers worth mentioning is Japanese composer Isao Tomita's \textit{Snowflakes Are Dancing}, which was released in 1974. The use of synthesizers in music and film extends into almost all genres of music and was the cornerstone in the development of new genres including techno and other electronic music genres. [Also note something about the historical context and other synthesizer manufacturers] -> Mark Jenkins provides an excellent overview of the use of synthesizers in various genres of music over the years in his book \textit{Analog Synthesizers: Understanding, Performing, Buying} \cite{jenkins2019analog}.

\subsubsection{Digital}
The first experiments with digit synthesis were conducted by Max Mathews on an IBM 704 computer in 1957 \cite{roads1980interview}. These experiments consisted of programming and synthesizing melodies using triangle waves, Mathew's program was able to accept the note pitch, amplitude, and duration. The Music III program was developed by Mathew's in 1960 and introduced an important concept called the \textit{unit generator}, which defined basic components of a synthesizer in programmatic units that a user was able to link together to create a full synthesizer architecture. Mathew's describes these concepts as being developed in parallel, but separately, to similar synthesis concepts (e.g. modular synthesizers) being developed in the analog world. He described this as "an advantage because a musician who knew who to patch together Moog synthesizer units would have a pretty good idea how to put together unit generators in the computer."

In 1973 John Chowning, a researcher at Stanford, released a landmark audio synthesis paper entitled on Frequency Modulation (FM) synthesis \cite{chowning1973synthesis}. The patent for FM synthesis was licensed to Yamaha who developed the Yamaha DX7 synthesizer using the technology, which was released in 1983 and became one of the best selling synthesizers of all time. One of the major benefits of FM synthesis is that it is able to produce complex audio waveforms at low computational cost. Additionally, the Yamaha DX7 was a fully polyphonic synthesizer, meaning that it was capable of producing multiple tones simultaneously (i.e. was able to play chords), whereas most analog synthesizers that were within the same time period were monophonic (only capable of playing one note at a time). The Yamaha DX7 was also incredibly difficult to program, although it came preloaded with a large selection of quality parameter settings, or presets, that allowed users to play the synth without have to learn how to program it.

As digital technology improved and computers became more powerful, new synthesis techniques such a sampling synthesis \cite{mcguire2015musical} and physical modelling \cite{jaffe1983extensions} emerged. Digital emulations of analog synthesizers, or virtual analog (VA) synthesis, also started to become popular in the 1990s. VA synthesizers attempt to express traditional analog synthesis methods in code as digital signal processing (DSP) algorithms. The development of more powerful computers also enabled recording workflows to be transferred into software and professional recording studios started to transition to digital with the release of Digidesign ProTools in the early 1990s. As software synthesis became more prevalent, many types of different software synthesizers emerged including VA software synthesis and a trend emerged of directly emulating analog equipment digitally, including the user interface. Skeumorphism describes computer user interfaces that attempt directly mimic their real-world counterpoint, and has become extremely common in audio software, including synthesizers; user interface researchers have begun to question whether or not these skeumorphic interfaces enhance usability of not \cite{lindh2018beyond}.

\subsubsection{Audio Plugins}
In 1996 Steinberg\footnote{\url{https://www.steinberg.net}} released the Virtual Studio Technology (VST) interface, which allowed third-party software including audio effects to be integrated into host applications including DAWs. Because VSTs integrate with host applications, they are also called audio plug-ins. The second version of VST was released in 1999 which added support for the Musical Instrument Digital Interface (MIDI) \cite{rothstein1992midi}, a communication protocol enabling musical hardware and software to exchange information and control signals. The addition of MIDI to the VST interface opened the doors for VSTi, VST instruments, including software synthesizers. Other audio plug-in architectures have been developed in addition to VSTs, and popular ones include Apple's Audio Units (AU) and Avid's Avid Audio eXtension (AAX). Audio plug-ins have allowed software developers a method to create unique audio effects and synthesizers, and an industry dedicated to their development has blossomed over the last few decades. At the time of writing there are over 500 different synthesizer plug-ins available for purchase or for free on the KVR\footnote{\url{https://www.kvraudio.com/plugins/softsynth-virtual-instruments}} database of audio products.

\subsection{Components of a Synthesizer}
Synthesizers can be viewed as comprising two major components: the synthesis engine, which is where sounds is generated, and a control interface which allows a user to control the synthesis engine. Audio synthesis can be a complex process so there is usually a high-level of abstraction between the synthesis engine and the control interface. The control interface presents a conceptual model of the synthesizer to the user and maps this model to the underlying engine. A control interface has a set of parameters that can be altered to modify the nature of the sound being generated. 

A skilled sound designer is able to interact with the control interface and craft sounds to fit the needs of their creative project. This process is referred to as programming a synthesizer. A well-designed synthesizer control interfaces allow users to build up a conceptual model that allows them to easily interact with the synthesizer in a way that allows them to be expressive. 

% Not so sure about this section.
% The nature of the control interface is guided by the synthesis method being used by the engine. Synthesis methods like subtractive synthesis which involve a clear linear signal flow that starts with a complex waveform that is progressively shaped can lead to a more simple conceptual model. Moog synthesizers are examples of subtractive synthesizers that have clear control interface that maps to the synthesis engine. More complex synthesis methods such as Frequency Modulation (FM) synthesis are more challenging to create clear control maps for. Interestingly, the most commercially successful synthesizer, the Yamaha DX7, had a notoriously difficult control interface, although shipped with an extensive high-quality set of factory presets (pre-defined control interface parameter settings).


\subsection{Synthesis Techniques}
- Sound Synthesis and Sampling provides a thorough overview of both analog and digital synthesizers and the various methods. \cite{russ2012sound}

\subsubsection{Neural Synthesis}
In contrast to traditional synthesis, neural synthesizers generate audio using large-scale machine learning architectures with millions of parameters \cite{engel2017neural}. Differentiable digital signal processing \cite{engel2020ddsp} bridged the gap between traditional DSP synthesizers with the expressiveness of neural networks, exploring a harmonic model-based approach, using a more compact architecture with 100K parameters.
One benefit of synthesized audio is that the underlying factors of variation ({\em i.e.}~the parameters) are known.

% GANs for synthesis
In this work, we use Generative Adversarial Networks (GANs) \cite{goodfellow2014generative} to generate new instrumental audio from a dataset of existing material. GANs have the potential to be used to generate new sounds on the fly. This would dramatically alleviate both the problem of having to pore through giant sound libraries, and the problem with having to only use one sample repeatedly. In addition, the explosion of new sounds which could potentially be produced by GANs would vastly reduce recording costs by designers of sound libraries.

This research avenue is to a certain degree untapped: GANs have been successfully applied to the generation and manipulation of images, however, relatively little work has been focused on the audio domain. Research related to the specific work proposed here was presented by \cite{donahue2018adversarial}  and \cite{engel2018gansynth}.

\subsection{Synthesizer Programming}
Both Carlos and Tomita excelled at patching synthesizer modules together and tuning the parameters of individual modules to create new sounds to orchestrate their performances. This process ... it's art! Blah blah blah. But also these folks are artists and virtuosos.

- Context on synthesizer programming \cite{jenkins2019analog}
- Russ describes the programming component intrinsically creative.
- When we talk about programming synthesizers we are referring to the task of selecting parameter settings for a synthesizer in order to achieve a desired sound
- Talk about the nature of this task. I think in that krekovic study there was something about this process? That this can be done in an exploratory way that ppl enjoy that process.
- Even so, it can still be quite a challenging task.

Programming audio synthesizers is challenging and requires a technical understanding of sound design to fully realize their expressive power. Traditional synthesizers can have over 100 parameters that affect audio generation in complex, non-linear ways. One of the most commercially successful audio synthesizers, the Yamaha DX7, was notoriously challenging to program. Allegedly nine out of ten DX7s coming into workshops for servicing still had their factory presets intact \cite{seago2004critical}.

Give an image of the complex synth UI.

