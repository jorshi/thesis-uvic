\section{Audio Synthesizers}
In the context of audio, a synthesizer refers to a device that generates sound or music. Martin Russ provides a thorough overview of audio synthesis in his textbook, Sound Synthesis and Sampling \cite{russ2012sound}. Russ introduces the synthesizer as any device that gerates sound,  even the human voice can be thought of as a synthesizer, however, sound synthesizers have become more broadly accepted as an electronic device that produces synthetic sounds. A synthesizer may do this through the playback and recombination of pre-existing audio material or through the generation of raw audio waveforms. There are numerous types of synthesis techniques that are capable of producing a huge variety of different sounds. Broadly speaking, these sounds can be categorized into two different classes: `imitative` or `synthetic`. Imitative sounds attempt to emulate a sound that exists in the natural world such as a physical musical instrument or a sound effect such as an explosion. Electric pianos are examples of imitative synthesizers. Synthetic sounds are those that have no relation to a sound in the physical world. The distinction between imitative and synthetic sounds is blurry and most sounds fall somewhere in the middle. For example, synth-brass sounds, which were a staple of synths such as Roland's popular Jupiter-8, are sounds that are based on an imitation of a brass sound but extend into the synthetic realm. 

% This is maybe a bit more like introduction material?
Synthesizers have become ubiquitous in audio production and entire genres of music have been developed around their use. Bebe and Louis Barron produced the first electronic film score for the movie "Forbidden Planet" in 1955 using synthetic sounds. Since then the use of synthesizers for movies and video games has also become common-place.


\subsection{Evolution of Synthesizers}
The Telharmonium instrument was developed by Thaddeus Cahill in the early 1900s and was the first, and largest, sound synthesizer \cite{roads1996computer}. On September 26, 1906, an audience of 900 individuals came to view the massive electronic instrument that was capable of producing pure sinusoidal waves in at frequencies in integer ratios. 

\subsubsection{Analog}
Until the late 1950s all synthesizers were analog. All analog synthesizers are defined by their use of continuous-time signals, as opposed to digital systems which operate in discrete chunks of information, or discrete-time signals. Early analog synthesizers can be broken down into two broad categories: (1) sounds that are generated directly by electric circuits by oscillating vaccuum tubes, or (2) rotating or vibrating physical systems that are controlled by electronic sources. 
- Theremin, Ondes Martenot,
- Hammond Novachord was the first fully polyphonic commercial synthesizer. 
- Buchla / Moog
- Some of the 



- \cite{mcguire2015musical} provides a good overview of the historical context for synthesizers 
- \cite{jenkins2019analog} provides an amazing overview of the history of a bunch of different manufacturers
- Buchla / Moog
- Waveforms

\subsubsection{Digital}
The first experiments with digit synthesis were conducted by Max Mathews on an IBM 704 computer in 1957 \cite{roads1980interview}. These experiments consisted of programming and synthesizing melodies using triangle waves, Mathew's program was able to accept the note pitch, amplitude, and duration. The Music III program was developed by Mathew's in 1960 and introduced an important concept called the \textit{unit generator}, which defined basic components of a synthesizer in programmatic units that a user was able to link together to create a full synthesizer architecture. Mathew's describes these concepts as being developed in parallel, but separately, to similar synthesis concepts (e.g. modular synthesizers) being developed in the analog world. He described this as "an advantage because a musician who knew who to patch together Moog synthesizer units would have a pretty good idea how to put together unit generators in the computer." 
- FM Synthesis - \cite{chowning1973synthesis}, lead to DX7
- anything else to note here?

\subsubsection{Hardware vs. Software}
Hardware synthesizers are physical devices purpose built for the task of audio synthesis. There come in many forms, but generally their is a distinction made between performance synthesizers and modular synthesizers. Performance synthesizers are contained units with predefined components that are linked together into a particular synth architecture.

\subsubsection{Audio Plugins}
In 1996 Steinberg\footnote{\url{https://www.steinberg.net}} released the Virtual Studio Technology (VST) interface, which allowed third-party software including audio effects to be integrated into host applications including digital audio workstations (DAWs). Because VSTs integrate with host applications, they are also called audio plug-ins. The second version of VST was released in 1999 which added support for the Musical Instrument Digital Interface (MIDI) \cite{rothstein1992midi}, a communication protocol enabling musical hardware and software to exchange information and control signals. The addition of MIDI to the VST interface opened the doors for VSTi, VST instruments, including software synthesizers. Other audio plug-in architectures have been developed in addition to VSTs, and popular ones include Apple's Audio Units (AU) and Avid's Avid Audio eXtension (AAX). Audio plug-ins have allowed software developers a method to create unique audio effects and synthesizers, and an industry dedicated to their development has blossomed over the last few decades. At the time of writing there are over 500 different synthesizer plug-ins available for purchase or for free on the KVR\footnote{\url{https://www.kvraudio.com/plugins/softsynth-virtual-instruments}} database of audio products.

\subsection{Components of a Synthesizer}
- Talk about the modular nature of synthesis? While there are many different types synthesis techniques, in audio processing it is common to talk about blocks, where each block is responsible for a unit of processing. Synthesizers can be thought of containing individual processing blocks as well that can be linked together in various configurations to build up a synthesis engine. Regardless of the type of synthesizer, there are several blocks that are common found in many synthesizers. [Listing of different types of synthesizer components]

Synthesizers can be viewed as comprising two major components: the synthesis engine, which is where sounds is generated, and a control interface which allows a user to control the synthesis engine. Audio synthesis can be a complex process so there is usually a high-level of abstraction between the synthesis engine and the control interface. The control interface presents a conceptual model of the synthesizer to the user and maps this model to the underlying engine. A control interface has a set of parameters that can be altered to modify the nature of the sound being generated. A skilled sound designer is able to interact with the control interface and craft sounds to fit the needs of their creative project. This process is referred to as programming a synthesizer. A well-designed synthesizer control interfaces allow users to build up a conceptual model that allows them to easily interact with the synthesizer in a way that allows them to be expressive. [Is their a citation here regarding flow?]. The nature of the control interface is guided by the synthesis method being used by the engine. Synthesis methods like subtractive synthesis which involve a clear linear signal flow that starts with a complex waveform that is progressively shaped can lead to a more simple conceptual model. Moog synthesizers are examples of subtractive synthesizers that have clear control interface that maps to the synthesis engine. More complex synthesis methods such as Frequency Modulation (FM) synthesis are more challenging to create clear control maps for. Interestingly, the most commercially successful synthesizer, the Yamaha DX7, had a notoriously difficult control interface, although shipped with an extensive high-quality set of factory presets (pre-defined control interface parameter settings).

- Oscillators

\subsection{Synthesis Techniques}
- Sound Synthesis and Sampling provides a thorough overview of both analog and digital synthesizers and the various methods. \cite{russ2012sound}

\subsection{Synthesizer Programming}
- Context on synthesizer programming \cite{jenkins2019analog}
- Russ describes the programming component intrinsically creative.
- When we talk about programming synthesizers we are referring to the task of selecting parameter settings for a synthesizer in order to achieve a desired sound
- Talk about the nature of this task. I think in that krekovic study there was something about this process? That this can be done in an exploratory way that ppl enjoy that process.
- Even so, it can still be quite a challenging task.

Programming audio synthesizers is challenging and requires a technical understanding of sound design to fully realize their expressive power. Traditional synthesizers can have over 100 parameters that affect audio generation in complex, non-linear ways. One of the most commercially successful audio synthesizers, the Yamaha DX7, was notoriously challenging to program. Allegedly nine out of ten DX7s coming into workshops for servicing still had their factory presets intact \cite{seago2004critical}.

Give an image of the complex synth UI.

\subsubsection{Neural Synthesis}
In contrast to traditional synthesis, neural synthesizers generate audio using large-scale machine learning architectures with millions of parameters \cite{engel2017neural}. Differentiable digital signal processing \cite{engel2020ddsp} bridged the gap between traditional DSP synthesizers with the expressiveness of neural networks, exploring a harmonic model-based approach, using a more compact architecture with 100K parameters.
One benefit of synthesized audio is that the underlying factors of variation ({\em i.e.}~the parameters) are known.

% GANs for synthesis
In this work, we use Generative Adversarial Networks (GANs) \cite{goodfellow2014generative} to generate new instrumental audio from a dataset of existing material. GANs have the potential to be used to generate new sounds on the fly. This would dramatically alleviate both the problem of having to pore through giant sound libraries, and the problem with having to only use one sample repeatedly. In addition, the explosion of new sounds which could potentially be produced by GANs would vastly reduce recording costs by designers of sound libraries.

This research avenue is to a certain degree untapped: GANs have been successfully applied to the generation and manipulation of images, however, relatively little work has been focused on the audio domain. Research related to the specific work proposed here was presented by \cite{donahue2018adversarial}  and \cite{engel2018gansynth}.

