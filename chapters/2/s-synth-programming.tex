\section{Synthesizer Programming}
 A skilled sound designer is able to interact with the control interface and craft sounds to fit the needs of their creative project. This process is referred to as programming a synthesizer. 

Both Carlos and Tomita excelled at patching synthesizer modules together and tuning the parameters of individual modules to create new sounds to orchestrate their performances. This process ... it's art! Blah blah blah. But also these folks are artists and virtuosos.

\subsection{Challenges}
- Challenges with synthesizer programming UI design: \cite{seago2013new}. -- "Some, like subtractive synthesis, offer controllers which are broadly intuitive, in that changes to the parameter values produce a proportional and predictable change in the generated sound. Other methods, however, are less easily understood. FM synthesis, for example, is a synthesis method that may be viewed as essentially an exploration of a mathematical expression, but whose parameters have little to do with real-world sound production mechanisms, or with perceived attributes of sound." -- There are some other really good bits about navigating the timbre space and non-linearities.
- \cite{seago2004critical} Thus, under most current systems, the user is obliged to express directives for sound specification in system terminology, rather than in language derived from the user domain. Three types of synthesizer interfaces: parameter selection in a fixed architecture, architecture specification and configuration, and direct specification of physical characteristics of sound.
- Context on synthesizer programming \cite{jenkins2019analog}
- Russ describes the programming component intrinsically creative.
- When we talk about programming synthesizers we are referring to the task of selecting parameter settings for a synthesizer in order to achieve a desired sound

- Talk about the nature of this task. I think in that krekovic study there was something about this process? That this can be done in an exploratory way that ppl enjoy that process. \cite{krekovic2019insights}
- 122 participants. 71\% had ten or more years of experience using synthesizers and only 2.7\% claimed they were novice users. Also, a majority of users had some formal training as musicians. So this study has quite a high representation of users who are experienced, as opposed to studying individuals who are novices. Musical training was only a weak indicator of experience using synthesizers. Most users most often create a synth patch from scratch or modify an existing one and only sometimes use presets with no modification. Interestingly there was no correlation with experience or music education. All users, regardless of experience or musical knowledge most often tweak the parameters of synthesizers.
- Four impediments to manually programming a synthesizer: 1) It can be time consuming, 2) it can distract them from focusing on music, 3) it can be difficult and non-intuitive to learn how to use a particular instrument, and 4) it rarely leads to desirable results. Participants agreed with the first three statements. Most participants disagreed with the fourth statement, but users with less experience were more likely to agree. Considering the average experience level of the group, these results highlight the challenges with synthesizer programming in the context of creating music.

- Even so, it can still be quite a challenging task.

Programming audio synthesizers is challenging and requires a technical understanding of sound design to fully realize their expressive power. Traditional synthesizers can have over 100 parameters that affect audio generation in complex, non-linear ways. One of the most commercially successful audio synthesizers, the Yamaha DX7, was notoriously challenging to program. Allegedly nine out of ten DX7s coming into workshops for servicing still had their factory presets intact \cite{seago2004critical}.

\subsection{Opportunities}
Krekovi\'{c} \cite{krekovic2019insights} participants were also asked to rate the perceived helpfulness of improvements to synthesizer  interfaces. Four different systems were proposed: 1) random presets are generated within a category, 2) a user provides a description of a the desired sound and a preset is generated for them, 3) a user provides an example sound and the system generates a presets to sound similar, and 4) more intuitive interactive user interface. Participants thought that proposed systems 3 and 4 would be helpful and systems 1 and 2 would be slightly helpful.
