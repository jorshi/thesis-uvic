\section{Synthesizer Programming}
Users generally have two options when trying to generate a desired sound on a synthesizer: they can build up the sound from scratch by adjusting parameters, or they can use a preset parameter setting designed by someone else. Many users also use presets as a starting point and manually adjust parameters to tune the sound \cite{krekovic2019insights}. Synthesizer programming refers to the process of manually adjusting parameters, whether starting from scratch or a preset. Programming a rich sound is considered an art \cite{russ2012sound}, and the quality of the sounds designed by Carlos and Tomita are one of the reasons their work achieved critical acclaim \cite{jenkins2019analog}. Specific techniques for programming synthesizers have lead to the production of sounds that define genres of music, especially electronic music, such as the "wobble bass" sound characteristic of dubstep. [What else can I add here to describe the importance of sound design? Is there anything else?]

\subsection{Challenges}
In the previous section on synthesizer control interfaces, \S\ref{sec:control_interfaces}, three different dimensions of sound were introduced: pitch, loudness, and timbre. Pitch and loudness are uni-dimensional and have relatively simple mappings to parameters \cite{seago2004critical}. The remaining parameters on a synthesizer, which is the vast majority of them, are used to specify the timbre of the resulting sound. This means that the primary task of programming a synthesizer is using a control interface to design the timbre of a sound. But what is timbre? The ANSI definition describes it as the attribute of auditory sensation that allows it to be distinguished from other sounds at the same pitch and loudness \cite{american1973american}. This definition doesn't tell us very much about what timbre is, as opposed to what it is not. Understanding precisely what musical timbre \textit{is} has presented itself as challenging problem \cite{krumhansl1989musical} and significant research has been conducted to try to answer this \cite{McAdams2019}. The spectrotemporal conception of timbre [cite Risset] describes timbre as a series of time-varying spectrums. Building on this, Grey introduced the concept of \textit{Timbre Space} \cite{grey1977multidimensional} as a multidimensional representation of timbre. Risset and Wessel explore timbre in the context of sound synthesis and emphasize the importance of considering the evolution of frequencies over time \cite{risset1999exploration}. 

- What can we draw from this: Programming a synthesizer is specifying how a spectrum evolves over time. It is impractical to do this directly considering the number of parameter values that would need to be set and the values themselves are not easily subjectively interpretable. Different synthesis techniques provide methods for accessing vast regions of the space possible timbres using a more compact set of parameters. 

The complexity of defining what timbre is and how we can accurately represent it highlights one of the challenges of synthesizer programming: users are tasked with navigating a complex multidimensional perceptual space that doesn't have clearly defined boundaries or axes. 

Synthesizer control interfaces provide the means for navigating this space. Most manufacturers avoid venturing into the vague territory of providing perceptually motivated controls and instead allow the underlying synthesis engine to dictate the design \cite{seago2004critical}. As a result, timbre control is mostly specified by a large number of numerical parameters that are unique to the synthesis technique \cite{ethington1994seawave}. Allan Seago describes some of the challenges in programming synthesizers based on various synthesis techniques \cite{seago2013new}:

\begin{quote}
    Some [synthesis techniques], like subtractive synthesis, offer controllers which are broadly intuitive, in that changes to the parameter values produce a proportional and predictable change in the generated sound. Other methods, however, are less easily understood. FM synthesis, for example, is a synthesis method that may be viewed as essentially an exploration of a mathematical expression, but whose parameters have little to do with real-world sound production mechanisms, or with perceived attributes of sound.
\end{quote}

Both Wendy Carlos and Isao Tomita were both masters at programming synthesizers. They learned the connection between the parameters of the synthesizers and how to interconnect modules to create the rich timbres that made their music so special. This involved developing a deep understanding of the technical details of the synthesis engine, how the parameters affected synthesis, and how to proficiently navigate the timbre space of their synthesizers. A recent study conducted by Gordan Krekevi\'{c} sought to elucidate how a more general population of synthesizer users \cite{krekovic2019insights}. 122 individuals participated in that study, which consisted of an online survey with questions regarding their experience with synthesizers. A majority of users were very experienced with synthesizers, 71\% had ten or more years of experience, and only 2.7\% were novice users, having less then a few months of experience. Additionally, a majority of users had some formal training as musicians, although having formal musical training was only a weak indicator of experience using synthesizers. When asked about how they engaged with their synthesizers users claimed that they would most often program sounds from scratch or modify a preset as opposed to using a preset with no modification. Interestingly, there was no correlation with experience or level of music education; all users, regardless of their background experience, most often engaged in the manually tweaking parameters of their synthesizers to create sounds. 

Krekov\'{c} identified four impediments to manually programming a synthesizer:
\begin{enumerate}
    \item it can be time consuming,
    \item it can be a distraction from focusing on music,
    \item it can be difficult and non-intuitive to learn to use a particular instrument,
    \item it rarely leads to desirable results.
\end{enumerate}

Most participants agreed with statements 1-3 and disagreed with statement 4, however, participants with less experience were more likely to agree with statement 4 and participants who more often programmed from scratch were less likely to agree. This indicates that users with more experience programming synthesizers more often felt that they were able to achieve desirable results. When given the opportunity to write about their experiences in a more open-ended manor participants generally reported on difficulties with user-interfaces, learning specific synthesizers, limited features, and the creative process. One participants' comments highlight some of the challenges and commitment associated with learning to use synthesizer.

\begin{quote}
    Different sorts of synthesis require different background knowledge, most of which have steep learning curves that are at least partially exclusive. In other words, there is an enormous investment of time to deeply learn how the different forms of synthesis work. This learning is a prerequisite to effective use of synthesizers.
\end{quote}

\subsection{Opportunities}
Considering that well over half of the participants in Krekovi\'{c}'s study had over ten years of experience, the results highlight the degree of difficulty associated with programming synthesizers and the opportunity for improvements. Finding alternative and assistive methods for synthesizer programming has been an active field of study for over 40 years and will be reviewed in detail in the following chapter. Based on a review of some of the methods from previous research, Krekovi\'{c} selected four different methods and asked participants to rate the perceived helpfulness of each system. The systems proposed were: 1) a system that generates random presets within a category, 2) a user provides a description of a the desired sound and a preset is generated for them, 3) a user provides an example sound and the system generates a presets to sound similar, and 4) more intuitive interactive user interface. Participants thought that proposed systems 3 and 4 would be helpful and systems 1 and 2 would be slightly helpful. 

Due to the limited number of novices included the study, it is challenging to to say with certainty how these results would generalize to users in that category, and this would be a good area for future work. However, based on the results that showed a correlation between experience and satisfaction with results of programming, there is likely a good opportunity to improve the experience of novice users.

\subsection{Enjoyment}
When conducting a critical analysis on the usability of synthesizers and the affect that has on synthesizer programming, it is easy to take on the perspective that it is a strictly negative trait that synthesizers are challenging to use and that is something that must be improved. There is, however, some indication that users enjoy the complexity of synthesizer interfaces and the challenge that provides in their programming. This is illustrated by the recent resurgence in hardware modular synthesizers \cite{bates2021interface}. Modular synthesizers are more complicated and time-consuming to use than software synthesizers for a number of number of reasons, and are more likely to lead to unsatisfactory results, yet the demand for them has shot up in recent years. This points to realization that programming a synthesizer, despite it being challenging and time-consuming, is an enjoyable experience and something worth doing regardless of the quality of the final result. Results from Krekovi\'{c}'s study point to this conclusion as well, despite the fact that participants agreed with three out of four of the impediments, they still chose to manually program more often than not, and most had for many years. This conclusion could of course be drawn based on survivor bias, the users who participated in the study hadn't given up with synthesizer programming in the face of the challenges; they either enjoyed it enough, or had another good reason, to keep going despite the impediments. How many potential synthesizer users are not represented because they never made it past the initial learning curve? While it is arguable whether using a musical instrument should be easy \cite{mcdermott2013should}, and that long term enjoyment may depend on the existence of challenges and idiosyncrasies, there is most certainly an opportunity to improve and support the enjoyment of synthesizer programming at all stages of a users' process.


% - \cite{bates2021interface} - looks at interfaces of modular synthesizers (maybe better in control interfaces?) 
% - \cite{d2016interface} interface of software synthesizers and smart phones?

% A skilled sound designer is able to interact with the control interface and craft sounds to fit the needs of their creative project. This process is referred to as programming a synthesizer. 

% Both Carlos and Tomita excelled at patching synthesizer modules together and tuning the parameters of individual modules to create new sounds to orchestrate their performances. This process ... it's art! Blah blah blah. But also these folks are artists and virtuosos.

% - Challenges identified in \cite{ethington1994seawave}: Timbre is most commonly specified by a large collection of numerical parameters peculiar to a particular synthesis technique."
% - Challenges with synthesizer programming UI design: \cite{seago2013new}. -- "Some, like subtractive synthesis, offer controllers which are broadly intuitive, in that changes to the parameter values produce a proportional and predictable change in the generated sound. Other methods, however, are less easily understood. FM synthesis, for example, is a synthesis method that may be viewed as essentially an exploration of a mathematical expression, but whose parameters have little to do with real-world sound production mechanisms, or with perceived attributes of sound." -- There are some other really good bits about navigating the timbre space and non-linearities.
% - \cite{seago2004critical} Thus, under most current systems, the user is obliged to express directives for sound specification in system terminology, rather than in language derived from the user domain. Three types of synthesizer interfaces: parameter selection in a fixed architecture, architecture specification and configuration, and direct specification of physical characteristics of sound.
% - Context on synthesizer programming \cite{jenkins2019analog}
% - Russ describes the programming component intrinsically creative.
% - When we talk about programming synthesizers we are referring to the task of selecting parameter settings for a synthesizer in order to achieve a desired sound
