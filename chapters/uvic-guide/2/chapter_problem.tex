\startchapter{The Problem to be Solved}
\label{chapter:problem}

\newlength{\savedunitlength}
\setlength{\unitlength}{2em}

Here is where you tell me what is the problem you have been working on for the past few months (or years). I want all the details and you should not be timid about being too tutorial, except that you do not want to cross the line towards writing a textbook. However consider carefully that \textit{communication} implies conveying ideas to other people, while \textit{effective communication} occurs when your message is clearly understood. Remember that your audience must understand your message before they can agree with you.

Ask yourself:
\textit{who is your audience?} You might think of your supervisor who knows everything and you want to impress with your knowledge. I think instead of the graduate students who will be reading this thesis which is, after all, a property of the university. It is published as a university technical report so that others may learn by reading it. Then teach them! Be a bit tutorial. Even the expert external examiner will be impressed by your clarity of exposition if he or she does not need to read paragraphs twice in order to understand - something which people with PhDs and big egos find particularly irritating.

On the other hand, do not go too far and give trivial definitions from concepts learned in a 3rd year undergraduate courses, else you might find yourself in trouble when having to remember the details during an oral examination.

My approach is to put everything necessary to make clarity for
the problem the main goal of this chapter, assuming an intelligent and well prepared reader who already has a Bachelor degree in an appropriate subject.

Once I understand the problem clearly and its nuances (it may not be what I expected after all), I also need to know why the problem is important, what its impact is and what its application, if any. Here you are free to elaborate and write as much as you think is necessary to avoid the examination doubt that you have a brilliant new solution to a trivial and unimportant issue.

I suggest reading various books on how to do research and set up problems. The best for me was "The Craft of Research" by Wayne Booth \cite{booth1}, which can be found in the main library at Q180.55 M4B66. From there I have transferred to my writing a fairly simple structure for talking about the topic of the research, with the question to be asked and its motivation and significance. It goes as follows:
\begin{enumerate}
\item {\textit{I am trying to learn about (working on, studying...)}}
\item {\textit{because I want to find out....}}
\item {\textit{in order to understand...}}
\end{enumerate}

Another way of looking at this is to ask the
\textit{what}, \textit{why} and \textit{where}, starting from a \textit{setting} of the problem with a first point A, stating what the \textit{goal} is at point B and having an \textit{action link} between the two which will encompass your new solution. As surprising as this may be to some of you, I found reading a book from Microsoft very useful: "Beyond Bullet Points: Using Microsoft Office PowerPoint 2007 to Create Presentations That Inform" \cite {atkin}. The goal of the book is to improve presentations with Power Point, but there is a lot that can be transferred towards \textit{effective communication} for a thesis.

In summary, my view of the second chapter on
\textit{"The Problem to be solved"} is as follows:
\begin{enumerate}
\item {\textit{Not} all the background and definitions (boring!) - use instead just-in-time explanations as needed in every context as it comes up;}
\item {Motivation in depth;}
\item {Tutorial high level explanation, where it is important to choose the right pitch: who is the audience? who are you teaching here?}
\item {Make it exciting, make it current, make it important - why do I want to keep reading?}
\item {Should you list here the solutions from other researchers? I think not, list instead the different facets of the problems that other researchers have attacked.}
\item {A taxonomy can be extremely useful to place your problem and its particular special features within the perfect context of the overall area, as you need to make sure that the reader understands perfectly what you are trying to solve.}
\end{enumerate}


\setlength{\unitlength}{\savedunitlength}
