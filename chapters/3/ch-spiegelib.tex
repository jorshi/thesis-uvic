\graphicspath{{./}{./figures/}{./figures/spiegelib/}}

% \section{Goals}
% \begin{enumerate}
%     \item Main contributions: spiegelib software, initial experiments using Dexed.
%     \item Goal is to explore some of the main recent approaches to inverse synthesis and compare them in the task of finding parameters to match a target sound
%     \item Expand on the experimental research -- provide a bit more background on the specific methods that we are trying to emulate. Specifically: the genetic algorithm approaches and the deep learning approaches.
%     \item Identify some of the challenges faced with this approach: rendering from VSTs is slow. Recent techniques SPSA could allow for VSTs to be inserted directly into the pipeline. Leads us to an area for future work: torchsynth
% \end{enumerate}

% Will have given a bit of a history by this point in the background -- something similar to Brecht's paper, but for automatic synthesizer programming. Here we want to focus on the main techniques that we want to use.

% \section{Plan}
% \begin{enumerate}
%     \item This chapter explores the inverse synthesis problem. This is based on work I completed under the supervision of my supervisors and was published at AES ...
%     \item Contributes SpiegeLib software and demonstrates the steps of conducting an inverse synthesis experiment through some experiments.
%     \item Software in Inverse Synthesis -- overview the available software. Give some motivation based on reproducible research. Maybe add a figure that outlines the experimental process and how each of the library components fit together to help with this.
%     \item SpiegeLib
%     \item Experiment -- Go into more detail on the specific algorithms and methods that are being used here. Can give some code detail as well if that would be helpful.
%     \item Evaluation -- How are the results evaluated and list the results
%     \item Conclusion
% \end{enumerate}

\chapter{SpiegeLib: A framework for automatic synthesizer research}
\label{chapter:inverse_synth}
This chapter introduces SpiegeLib, a software framework for automatic synthesizer research that I developed under the supervision of my supervisors and published at the 148th Audio Engineering Society (AES) Convention in 2020 \cite{shier2020spiegelib}. SpiegeLib is an open-source software library written in the Python programming language with the goal of promoting collaboration and reproducibility in automatic synthesizer research. The development of the library was based on work published by Yee-King et al. \cite{yee2018automatic} that explored automatic synthesizer programming of a VST FM synthesizer using deep learning methods. In their work they focused on the inverse synthesis problem which has the goal of finding synthesizer parameters to match a target sound, also referred to as sound matching. SpiegeLib is designed to support research in inverse synthesizer and provide a platform for sharing and evaluating methods.

Vandewalle et al. argue that reproducibility in computational science research increases the impact of a work and they provide a framework for evaluating the quality of reproducibility \cite{vandewalle2009reproducible}. The aim of \mintinline{python}{spiegelib} is to provide a platform for researchers of automatic synthesizer programming to develop, test, and share implementations in a way that promotes reproducibility at the highest level. \mintinline{python}{spiegelib} stands for Synthesizer Programming with Intelligent Exploration, Generation, and Evaluation Library. The name \mintinline{python}{spiegelib} was chosen to pay homage to Laurie Spiegel, an early pioneer in electronic music composition. Laurie Spiegel is known for utilizing synthesizers and software to automate certain aspects of the music composition process. Her philosophy for using technology in music serves as a motivation for the \mintinline{python}{spiegelib} software library: "I automate whatever can be automated to be freer to focus on those aspects of music that can't be automated. The challenge is to figure out which is which." \cite{hinkle2006women}

\section{Software in ASP Research}
 In Vandewalle et al.'s paper on reproducibility in computational sciences, they advocate providing other researchers with "all the information (code, data, schemes, etc.) that was used to produce the presented results"\cite{vandewalle2009reproducible}. Several authors of ASP research have started to make their work open-access with source code available online. 
 
 Martin Roth and Matthew Yee-King developed $JVstHost$, a Java-based Virtural Studio Technology (VST) plugin host that was published by Matthew Yee-King \cite{yee2011automatic} and was a component of $SynthBot$ \cite{yee2008synthbot}. However, the code for $SynthBot$ itself was not released. Matthew Yee-King also shared the source code for $EvoSynth$, an application for interactive synthesizer patch exploration \cite{yee2016use}. A version of $EvoSynth$ is hosted online allowing for immediate experimentation\footnote{\url{http://www.yeeking.net/evosynth/}}. Krekovi{\'c} et al. released source code for their $MightyKnob$ system \cite{krekovic2016algorithm}. Esling et al. released open-source code and a Max4Live\footnote{\url{https://www.ableton.com/en/live/max-for-live/}} application for $FlowSynth$ \cite{esling2020flow}. Yee-King et al. recently took initial steps towards a software framework for ASP research with the release of source code that provides functionality for generating research datasets and a set of algorithms for parameter estimation \cite{yee2018automatic}. Along with that work they released the $RenderMan$\footnote{\url{https://github.com/fedden/RenderMan}} library for programmatically interacting with VST synthesizers using the Python programming language.
 
 \mintinline{python}{spiegelib} builds upon this work with the goal of supporting and encouraging reproducibility within the ASP research community. \mintinline{python}{spiegelib} is inspired by the steps that Yee-King et al. took towards creating a software library for ASP research and extends that work with the inclusion of: an object-oriented API, base classes for customization, more robust evolutionary techniques, basic subjective evaluation, complete documentation, and packaging and delivery. It provides a framework for authors to share implementations in an open-access way that allows other researchers to quickly recreate results using a clearly documented set of freely-available tools.
 
\section{Design of SpiegeLib}
\label{chapter:inverse_synth;section:spiegelib}

\mintinline{python}{spiegelib} is designed to be as extensible as possible to allow researchers to develop and test new implementations of components for conducting ASP research. There are several stages in a typical ASP experiment:

\textbf{1) Synthesizer configuration}: a synthesizer is selected and a subset of the parameters may be selected for estimation. For example, in work by Yee-King et al. \cite{yee2018automatic}, several different experiments were conducted using successively larger parameter subsets to increase the difficulty.

\textbf{2) Dataset generation}: for experiments requiring training, such as learning deep learning models, a dataset of synthesized audio and parameter pairs must be generated. Audio features may be extracted at this point too, which will be used as input to a model.

\textbf{3) Training models}: deep learning models are trained using the generated dataset.

\textbf{4) Sound matching}: this is the stage where parameters are estimated to match a target sound. In the case of deep learning models this means just inferring parameters using the trained model. For search techniques, the algorithm is run using the target audio as input.

\textbf{5 Evaluation}: results of the sound matching are evaluated here. Objective evaluation can be performed directly on the parameters as was the case in work by Barkan et al. \cite{barkan2019inversynth},  or audio can be rendered using the predicted parameters and evaluation carried out on the audio, which was done by both Barkan et al. and Yee-King et al. \cite{yee2018automatic}. Subjective evaluation can also be carried out at this point with a user listening experiment.

SpiegeLib contains components to support all stages of this experimental pipeline. Implementation details and the components of the library are detailed in the following section.

% - Figure for the research pipeline?

\section{Library Components}

Base classes with functionality for interacting with software synthesizers, audio feature extraction, parameter estimation, and evaluation provide an API to support development of custom implementations that will work with other components of the library. A number of utility classes are also provided for handling audio signals, generating datasets, and running experiments.

 \mintinline{python}{spiegelib} is written in the Python programming language and utilizes Python packages common in research including \mintinline{python}{numpy}, \mintinline{python}{scipy}, \mintinline{python}{tensorflow}, and \mintinline{python}{librosa}. \mintinline{python}{spiegelib} itself is a python package and is available through the Python Package Index (PyPI) with pip\footnote{\url{https://pypi.org/}}. All dependencies, except for \mintinline{python}{librenderman}, are python packages available through the PyPI and will be automatically installed by pip. For more information on installation, system requirements, and detailed library documentation, please refer to the  online documentation.\footnote{\url{https://spiegelib.github.io/spiegelib/}}
 
 A summary of the currently implemented algorithms is shown in table \ref{table:spiegel_algorithms}. A brief overview of these components and the main classes and functionalities of \mintinline{python}{spiegelib} is provided in the following sections.
 
 \begin{table*}[t]
\centering\small
\begin{threeparttable}
	\caption{Algorithms currently implemented as classes in \mintinline{python}{spiegelib}}
	\label{table:spiegel_algorithms}
	\begin{tabular}{|c|c|c|}
	\hline
	\multicolumn{3}{|c|}{\textbf{Algorithms in \mintinline{python}{spiegelib}}} \\
	\hline
	\hline
	\textbf{Feature Extraction} & \textbf{Deep Learning Estimators} & \textbf{Optimization Estimators} \\
	\hline
	FFT 		& MLP \cite{yee2018automatic} 		& Basic GA   			\\
	STFT 		& LSTM \cite{yee2018automatic} 		& NSGA III \cite{tatar2016automatic}				\\\cline{3-3}
	MFCC		& LSTM++ \cite{yee2018automatic} 	& \textbf{Objective Evaluation} 	\\\cline{3-3}
	Spectral\tnote{1}	& Conv6 \cite{barkan2019inversynth} 	& MFCC Evaluation 		\\
	\hline
	\end{tabular}
	\begin{tablenotes}[para, flushleft]
			\footnotesize
			\item[1] Spectral bandwidth, centroid, contrast, flatness, and rolloff.
	\end{tablenotes}
\end{threeparttable}
\end{table*}
 
\subsection{AudioBuffer}
The \mintinline{python}{AudioBuffer} class is used to pass audio signal signals throughout the library. It holds an array of audio samples and sample rate information. Methods of the \mintinline{python}{AudioBuffer} class provide functionality for loading audio from a variety of file formats, resampling, normalizing, time segmenting, plotting spectrograms, and saving audio as WAV files.

\subsection{Synthesizers}
The \mintinline{python}{SynthBase} class is an abstract base class that provides an interface for creating programmatic interactions with software synthesizers. \mintinline{python}{SynthBase} stores information and contains methods required for interaction with other components in \mintinline{python}{spiegelib}, including getting parameter lists, setting and getting patch configurations, overriding/freezing parameters, triggering audio rendering using MIDI notes, getting audio samples as \mintinline{python}{AudioBuffer}s, and requesting randomized patch settings. All patch settings are stored as a list of parameter tuples which contain the parameter number and parameter value. All parameter values are expected to be floating point numbers in the range [0.0, 1.0]. No requirement is made on how underlying synthesis engines are implemented, however, inheriting classes must provide parameter descriptions in a class attribute during construction and must provide implementations for four abstract class methods related to loading patches, randomizing patches, rendering audio, and returning an \mintinline{python}{AudioBuffer} of rendered audio.

%
% \mintinline{python}{load_patch()}, \mintinline{python}{randomize_patch()}, \mintinline{python}{render_patch()}, and \mintinline{python}{get_audio()}.

\mintinline{python}{SynthVST} is an implementation of \mintinline{python}{SynthBase} and provides an interface for interacting with VST synthesizers. \mintinline{python}{SynthVST} is a wrapper for the $RenderMan$ Python library developed by Leon Fedden in conjunction with research by Yee-King et al. \cite{yee2018automatic}.

\subsection{Audio Feature Extraction}
The abstract base class \mintinline{python}{FeaturesBase} provides an interface for audio feature extraction tasks. 
The \mintinline{python}{getFeatures()} abstract method must be overridden in inheriting classes and is where feature extraction algorithms are run.  \mintinline{python}{FeatureBase} also includes functionality for normalizing results from feature extraction. By default, data is normalized by removing the mean and scaling to unit variance. Settings for normalization can be saved from a set of data, reloaded, and applied to new feature extraction results to ensure that normalization is carried out using the same parameters. Currently, implemented feature extraction classes utilize the \mintinline{python}{librosa} library \cite{mcfee2015librosa} and include Mel Frequency Cepstral Coefficients (\mintinline{python}{MFCC}), Short Time Fourier Transform (\mintinline{python}{STFT}), Fast Fourier Transform (\mintinline{python}{FFT}), and a set of time summarized spectral features (\mintinline{python}{SpectralSummarized}).

\subsection{Datasets}
The \mintinline{python}{DatasetGenerator} class provides functionality for creating datasets of audio samples, feature vectors, and associated parameter settings from a synthesizer. An implementation of \mintinline{python}{SynthBase} and \mintinline{python}{FeaturesBase} are passed in as arguments to the \mintinline{python}{DatasetGenerator} constructor. To generate a dataset, random patches for the synthesizer are created and feature extraction is performed on the resulting audio. In this way, datasets for training and validating deep learning models, as well as datasets for evaluating sound matching experiments can be automatically generated. 
%The term contrived in the context of ASP research refers to the use of audio samples produced by the synthesizer being studied as a target for sound matching \cite{justice1979analytic}. The benefit of using a contrived sound for evaluation is that it minimizes the uncertainty regarding whether the synthesizer in question is capable of producing the target sound. 
External datasets can also be used within \mintinline{python}{spiegelib} and the \mintinline{python}{AudioBuffer} class provides support for loading folders of audio samples for processing.

\subsection{Estimators}
All parameter estimation classes implement the \mintinline{python}{EstimatorBase} abstract base class. \mintinline{python}{EstimatorBase} is a minimal base class with one abstract method, \mintinline{python}{predict()}, that has an optional input argument. Implementations of estimators are split into deep learning approaches and other approaches including evolutionary algorithms.  The included algorithms do not represent a comprehensive set of methods for ASP research but are meant to cover common methods informed by previous work. Six estimators are currently implemented and I plan to add 5 more in the near future: a hill climbing optimizer \cite{yee2018automatic}, a particle swarm optimizer \cite{heise2009automatic}, additional configurations of 2D CNNs \cite{barkan2019inversynth}, a 1D CNN for raw audio input \cite{barkan2019inversynth}, and a recent generative approach \cite{esling2020flow}.

\subsection{Deep Learning Estimators}
All deep learning models are implementations of the \mintinline{python}{TFEstimatorBase} abstract base class which utilizes the \mintinline{python}{tensorflow}\footnote{\url{https://www.tensorflow.org}} and \mintinline{python}{keras}\footnote{\url{https://www.tensorflow.org/guide/keras}} machine learning libraries. \mintinline{python}{TFEstimatorBase} implements \mintinline{python}{EstimatorBase} and provides wrapper functions for setting up data for training and validation, training models, running predictions, and saving and loading model weights. While these methods are designed to help in handling of data typical to a synthesizer parameter estimation problem, all methods for a \mintinline{python}{tf.keras.Model} can be accessed directly from the \mintinline{python}{model} class member. Classes that inherit from \mintinline{python}{TFEstimatorBase} define models in an implementation of the \mintinline{python}{buildModel()} method which is automatically called during construction in the base class. This allows new models to be quickly designed, switched out, and compared with minimal effort.

The currently implemented deep learning models are based on prior work, specifically on work on Recurrent Neural Networks by Yee-King et al. \cite{yee2018automatic} and work on Convolutional Neural Networks by Barkan et al. \cite{barkan2019inversynth}. For these works implementations for a Multi-Layer Perceptron (\mintinline{python}{MLP}), Long Short Term Memory (\mintinline{python}{LSTM}), Bi-directional Long Short Term Memory with Highway Layers (\mintinline{python}{HwyBLSTM}), and a convolution network with 6-layers (\mintinline{python}{Conv6}) are included. An example code listing of sound matching using a trained LSTM model is shown in figure \ref{fig:lstm_code}.

To save training and validation progress, the \mintinline{python}{TFEpochLogger} class can be passed in as a callback during model training. \mintinline{python}{TFEpochLogger} stores training accuracy and loss, and validation accuracy and loss over training epochs in a dictionary object which can be plotted after training.

%Example code listing
\begin{figure}[t]
\centering
\caption{Example of \mintinline{python}{spiegelib} performing a sound match from a target WAV file on a VST synthesizer. A pre-trained LSTM deep learning model is used with MFCC input.}
\footnotesize
\begin{minted}{python}
import spiegelib as spgl
import spiegelib.estimator.TFEstimatorBase

# Load VST and set parameters from JSON file
synth = spgl.synth.SynthVST('./Dexed.vst')
synth.load_state('./dexed_simple_fm.json')

# MFCC Audio Feature Extractor
ftrs = spgl.features.MFCC(normalize=True)

# Load saved normalization parameters
ftrs.load_normalizers('./normalizers.pkl')

# Load LSTM model from saved model file
lstm = TFEstimatorBase.load('./fm_lstm.h5')

matcher = spgl.SoundMatch(synth, lstm, ftrs)

target = spgl.AudioBuffer('./target.wav')
output = matcher.match(target)
output.save('./lstm_predicted_audio.wav')
\end{minted}
\label{fig:lstm_code}
\end{figure} 

\subsection{Optimization Estimators}
Two optimization estimators are currently implemented and utilize the DEAP python library \cite{fortin2012deap}. A basic GA (\mintinline{python}{BasicGA}) is included as well as a multi-objective non-dominated sorting genetic algorithm III (\mintinline{python}{NSGA3}). Both GAs require feature extraction objects, or a list of feature extraction objects in the case of the multi-objective algorithm, which are used in the GA evaluation function. 

\subsection{Evaluation}
Objective evaluation of results can be carried out by measuring error between audio samples. \mintinline{python}{EvaluationBase} is an abstract base class for calculating evaluation metrics on a set of target and prediction data. A list of target values and lists of predictions for each target are passed into the constructor. \mintinline{python}{EvaluationBase} provides functionality for calculating statistics on results, saving results as a JSON file, plotting results in histograms, and calculating metrics including mean absolute error, mean squared error, euclidian distance, and manhattan distance. Inheriting classes must implement the \mintinline{python}{evaluate_target()} method which is called for each target and associated estimations and is expected to return a dictionary of metrics for each estimation. The \mintinline{python}{MFCCEval} class implements \mintinline{python}{EvaluationBase} and calculates metrics on MFCC vectors for targets and estimations.

Functionality for conducting subjective evaluation of results is provided in the \mintinline{python}{BasicSubjective} class. This class accepts a set of audio files and runs a locally hosted server that generates a simple web interface for listening to and ranking audio files in terms of similarity or preference. For sound matching experiments, audio targets can be passed in along with a set of predictions for each target, and a sound similarity test will be generated with options for randomizing the ordering of targets and predictions. Results can then be saved as a JSON file.
%TODO: Add in a visual of this? What did we use to create this?

\subsection{Helper Classes}
- What other classes are available in SpiegeLib that we should comment on?

mintinline{python}{SoundMatch} is a functional class that uses an estimator to predict synthesizer parameter settings for an implementation of \mintinline{python}{SynthBase} in order to match a target sound.


\section{Future Work and Conclusion}
Development of \mintinline{python}{spiegelib} is ongoing and a number of expansions to the current library are planned. First, we would like to continue to expand the number of estimators available and plan on integrating the following: a hill climbing optimizer \cite{yee2018automatic}, a particle swarm optimizer \cite{heise2009automatic}, more 2D CNN configurations \cite{barkan2019inversynth}, a 1D CNN for raw audio input \cite{barkan2019inversynth}, and a generative approach \cite{esling2020flow}.  Second, we would like to expand on the type of interactions available such as automatic programming from vocal imitations \cite{mcartwright2014} and interactive methods. 
%Third, we would like to integrate more sophisticated subjective evaluation tools such as creating links to setup tests using the Web Audio Evaluation Tool \cite{waet2015}. Third, we would like to contribute to the $RenderMan$ library that is used in this work in order to extend support to Windows users and distribute the library via PyPI so the entire \mintinline{python}{spiegelib} library can be installed in one command. 
Finally, we would like to encourage developers and researchers from the automatic synthesizer programming community to contribute to \mintinline{python}{spiegelib}. Information on contributing is available online.\footnote{\url{https://spiegelib.github.io/spiegelib/contributing.html}} 

This work has introduced \mintinline{python}{spiegelib}, an open-source automatic synthesizer programming library. \mintinline{python}{spiegelib} is an object-oriented library that was designed with the goal of supporting development, collaboration, and reproducibility in the field. The library includes implementations of classes for conducting ASP research. These classes contain functionality for interacting with VST synthesizers, extracting audio features, creating datasets, estimating synthesizer parameters, and evaluating results. Six implementations of deep learning and evolutionary parameter estimation techniques based on previous work are included, with more planned.