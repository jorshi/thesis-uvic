\section{Creativity Support}
The field of creativity support is focused on the development of tools to enable and enhance the creative output of an individual or group, both in novice and expert users. Creativity support tools (CSTs) \cite{shneiderman2007creativity} span a wide array of application domains including visual art, textiles, cooking, and music. A central question that CSTs ask is: 
\begin{quote}
    "How can designers of programming interfaces, interactive tools, and rich social environments enable more people to be more creative more often?"
\end{quote}
 Shneiderman \cite{shneiderman2007creativity} outlines a set of design principles for developing creativity support tools which include: support exploratory search, enable collaboration, provide rich history keeping, design with low thresholds, high ceilings, and wide walls. In subsequent related work, Davis \textit{et al.} focus on the role that CSTs play in supporting novices engaging in creative tasks and the relationship that the environment plays in creativity \cite{davis2013toward}. In their work, the authors identify two types of novice users: domain novices and tool novices. Domain novices are new to both the creative domain as well as using the creativity support tool. Tool novices have experience with the creative domain, but are novices at using a particular tool. To help evaluate and promote the development of creativity support tools for novices, they also propose a theory of creativity support based on cognitive theory.

These concepts provide an important platform for beginning to develop tools to support users of synthesizers. Both types of novices described by Davis \textit{et al.} are common and serve to benefit from the development of improved methods for interacting with them; domain novices are both new to sound design / music production as well as to using a specific synthesizer, whereas a tool novices would likely have experience with sound design / music production, but would be a novice with using a specific synthesizer. 

\subsection{Music Information Retrieval}
The field of music information retrieval (MIR) is a growing research area that was born out of the need to navigate increasingly large collections of digital music. Creative MIR is a subset of the field that is focused on applying the techniques from MIR towards creative applications including music production \cite{humphrey2013brief}.

- Conversations with music producers: \cite{andersen2016conversations}

- More directly related to music the fields of MIR (creative MIR) as well as intelligent music production. - automatic mixing ref, other topics (hit up some references and expand on this)

- Overview of approaches in intelligent music production \cite{moffat2019approaches}, automatic mixing \cite{de2017ten}.

- MIR audio querying: Automatic synthesizer programming can be viewed as a retrieval task as well as an optimization problem. Viewed as a retrieval task, the problem is similar to the MIR query tasks such as Query-by-example \cite{zloof1977query}, Query-by-vocal-imitation \cite{blancas2014sound}, and query-by-beat-boxing \cite{kapur2004query}. Query problems generally build up a model of the synthesis parameter space and then return a parameter setting based on a classification that attempts to match the input with the best parameter setting.
- Visualizing sounds: \cite{wessel1979timbre} (Potentially add the George citation on visualizing sounds).

- \cite{pardo2019learning} "Learning to build natural audio production interfaces" -> Rather than force nonintuitive interactions, or remove control altogether, we reframe the controls to work within the interaction paradigms identified by research done on how audio engineers and musicians communicate auditory concepts to each other: evaluative feedback, natural language, vocal imitation, and exploration